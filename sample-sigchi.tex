\documentclass[sigchi]{acmart}

\AtBeginDocument{%
  \providecommand\BibTeX{{%
    \normalfont B\kern-0.5em{\scshape i\kern-0.25em b}\kern-0.8em\TeX}}}

\settopmatter{printacmref=false, printccs=false, printfolios =false}
\setcopyright{none} 
\renewcommand\footnotetextcopyrightpermission[1]{}

\acmConference[SDCC Project A.A. 2018-2019]{ }{September 18, 2019}{ }

\begin{document}

\title{Byzantine Fault-Tolerant and Locality-Aware Scheduling MapReduce}

\author{Andrea Graziani}
\email{andrea.graziani93@outlook.it}
\affiliation{%
  \institution{Università Degli Studi di Roma Tor Vergata}
  \city{Rome}
  \state{Italy}
}

\renewcommand{\shortauthors}{Andrea Graziani (0273395)}

\begin{abstract}
\textit{MapReduce} is a programming paradigm that enables massive scalability across hundreds or thousands of servers allowing to process very large data-set \cite{IBMWhatIsMapReduce}.

Both academic literature and daily experience shows that \textit{arbitrary faults} frequently occur corrupting our results \cite{BFLMapReduce}. Moreover, ignore data-set locality can lead to performance degradation and a pointless bigger network traffic \cite{LARTS}.

In this paper we present an original MapReduce runtime system capable both to tolerate arbitrary faults and to recognize input data network locations and sizes in order to perform a \textit{locality-aware task scheduling}, improving performance and diminishing network traffic.

Although job's execution using our system requires more resources respect to other implementations, we believe that this cost is acceptable for critical applications that require an higher level of fault tolerance.
\end{abstract}

\keywords{MapReduce, Fault tolerance, Arbitrary failure, Data locality}
\maketitle

\section{Introduction}

Various data-intensive tasks, like seismic simulation, natural language processing, machine learning, astronomical data parsing, web data mining and many other, require a processing power that exceeds the capabilities of individual computers; this fact imposes the use of \textit{distributed computing}. Nowadays many famous distributed applications use thousands of computers and hundreds of other devices like network switches, routers and power units in order to provide their services to an increasing number of users in every part of the world, moving consequently an huge amount of data between computers and server. \textit{MapReduce}, a framework developed by Google, represents a solution for processing large data sets in a distributed environment. 

However, as many studies confirm, \textit{hardware component failures are frequent} and they will probably happen more often in the future owing to the increasing number of computer and server connected to internet. Is been documented that in the first year of a cluster at Google there were 1000 individual machine failures and thousands of hard drive failures \cite{PetaScaleFailure}. A recent study of DRAM errors in a large number of servers in Google data-centres for 2.5 years concluded that these errors are more prevalent than previously believed, with more than 8\% DIMM affected by errors yearly, even if protected by error correcting codes (ECC) \cite{DRAMError}. A Microsoft study of 1 million consumer PCs shown that CPU and core chipset faults are also frequent. \cite{MicrosoftStudyFailure} Moreover moving large amount of data repeatedly to distant nodes is becoming the bottleneck owing to an increased network traffic causing performance degradation.

These are the reasons why to construct a distributed system in such a way it can provide its services even in the presence of failures is become so critical; consequently, to provide a \textit{fault tolerant} cloud application represents an important goal in distributed-systems design. Moreover exploiting data locality, in order to mitigate network traffic and delay, becomes very important to improve performance.

Then the goal of this paper is to describe an \textit{Arbitrary Fault-Tolerant Locality-Aware} (AFTLA) \textit{MapReduce runtime system} capable to mitigate problems described above. 

\section{Arbitrary Fault-Tolerant Locality-Aware MapReduce}

\subsection{Arbitrary fault tolerance}

As known academic literature describes many type of failure, like \textit{crash failures}; however the most serious are known as \textit{arbitrary failures} or \textit{Byzantine failures}, according to which a server may produce arbitrary responses at arbitrary times which cannot be detected as being incorrect. 

Redundancy represents the key technique used to manage these kind of failure, according to which, 


The key approach to tolerating a faulty process is to organize several identical processes into a group. 


 Our BFT
MapReduce follows the approach of executing each task more
than once, similarly to the works mentioned above. The chal-



Process groups are part of the solution for building fault-tolerant systems.
In particular, having a group of identical processes allows us to mask one
or more faulty processes in that group. In other words, we can replicate
processes and organize them into a group to replace a single (vulnerable)
process with a (fault tolerant) group. 

An important issue with using process groups to tolerate faults is how
much replication is needed. To simplify our discussion, let us consider
only replicated-write systems. A system is said to be k-fault tolerant if it
can survive faults in k components and still meet its specifications. If the
components, say processes, fail silently, then having k + 1 of them is enough
to provide k-fault tolerance. If k of them simply stop, then the answer from
the other one can be used.

On the other hand, if processes exhibit arbitrary failures, continuing to
run when faulty and sending out erroneous or random replies, a minimum
of 2k + 1 processes are needed to achieve k-fault tolerance. In the worst case,
the k failing processes could accidentally (or even intentionally) generate the
same reply. However, the remaining k + 1 will also produce the same answer,
so the client or voter can just believe the majority.




\section{System Architecture}

In order to properly describe our MapReduce algorithm, including our arbitrary failures management system and how we exploit data locality in order to improve performance, it is necessary to describe our system architecture. 

\subsection{Assumptions}

Our system is composed by a set of distributed processes, every of which run on different hosts in a same data-center; from implementation point of view, every process run on his own Amazon EC2 server hosted in a same region.

We assume that our system runs \textit{asynchronously}, that is no assumptions about process execution speeds or message delivery times are made; therefore we can normally use timeouts to conclude that a process has crashed but, occasionally, such conclusion is false. However, all processes are connected by \textit{reliable channels}, so no messages are lost, duplicated or corrupted; that feature is guaranteed by the use of TCP connections. 

Clients are always correct, because if they are not there is no point in worrying about the correctness of our system's output.

finally we assume the existence of an hash function that is \textit{collisions-resistant}, for which it is infeasible to find two inputs that produce the same output.

\subsection{System's processes}

Our system is made up of three type of process:

\begin{description}
\item[Client Process] Which requests the execution of jobs composed by map and reduce tasks among

\item[Primary Process] Similarly to \textit{JobTracker} process used in Apache Hadoop, its duty is to satisfy clients requests, scheduling \textit{map} and \textit{reduce} tasks, coordinating at the same time \textit{worker processes} activities. 

\item[Worker Process] It executes map or reduce task scheduled by current primary process.
\end{description}

Unlike Hadoop, according to which the \textit{JobTracker} process is assumed always correct, the host where primary process is running may fail, for example by crashing or by losing network connectivity; in other words, primary process's host represents a \textit{single point of failure}. Therefore, to ensure an high system availability, we have adopted an architecture based on multiple primary process copies run on different host, one of which, using a leader election algorithm, is elected as system coordinator. When current leader fails, a backup copy is promoted to become the new coordinator. From an implementation point of view, we used services offered by Apache ZooKeeper to implement our leader election mechanism.


In order to achieve its duty, current primary process leader stores various information about requests received by clients, like their status and other informations about worker processes activities. Is very important to specify that such data are stored in memory, therefore they are permanently lost after a crash; this design makes our system  easier to implement and helps to reducing overhead due to the disk I/O activities. However recover lost in-memory leader state after a failure is required in order to satisfy clients requests. During each step of MapReduce framework, current primary leader process save current client requests status using an external fault tolerant services, in our case Apache zookeeper. When a primary process backup becomes leader, it retrieves all data from Apache ZooKeeper, restarting all pending client request from last saved state.

Current leader primary process can interact with worker processes using a push-based approach in order to schedule map or reduce tasks.






\subsection{The algorithm}

To make our system arbitrary fault-tolerant we have followed the approach according to which each task is executed more than once by different processes running on different host. This design requires to organize several identical processes into a group through which became possible to mask one or more faulty processes. 

As known, MapReduce framework splits input file in several splits to each of which a map task is associated; in other words, if an input file is split, for instance, into $n$ elements, primary process has to schedule $n$ map tasks. In order to achieve arbitrary fault-tolerance, each map task has to be executed by a groups of identical process performing given task. According to this design, in order to manage $n$ map tasks, $n$ different groups of processes are needed; these group are called Worker Groups. A worker group is a set of equal worker processes, each of which execute the same commands using same input data in the same order. In a group all worker processes run independently on different host and they do not interact with each other in any way. Current Leader Primary Process can interact with groups members using a push-based approach in order to schedule map or reduce tasks. This is the reason according to which this design is more expensive than using the original MapReduce runtime or Hadoop.

To be more precise, suppose to have $n$ worker groups and to want tolerate at most $k$ faulty processes for each group. To achieve arbitrary fault-tolerance, we can apply, for instance, a consensus algorithm like Paxos or Raft in order to reach consensus among all processes belonging to a given worker group. However that solution is too expensive because it requires $3f + 1$ replicas; moreover if we have, for instance, $n$ worker groups, we need at least $n(3k + 1)$ processes.

This is the reason according to which we have adopt an 
An important issue with this design is how much replication is needed. As known, if processes exhibit arbitrary failures, continuing to run when faulty and sending out erroneous or random replies, a minimum of $2k+1$ processes are needed to achieve $k$-fault tolerance. In the worst case, the $k$ failing processes could accidentally generate the same reply. However, the remaining $k+1$ will also produce the same answer, so the primary process just believe the majority. Notice that, for instance, applying consensus algorithm to reach consensus among k group members may suffer from fail-arbitrary failures requires 3f + 1 replicas to tolerate at most f faulty replicas.

 we have discarded this option considering it too expensive through which

,

In replicated-write protocols, an update is forwarded to several replicas at the same time. 


  we have a group of 3k+1 processes. Our goal is to show that we can establish a solution in which k group members may suffer from fail-arbitrary failures, yet the remaining nonfaulty processes will still reach consensus


All worker processes are running are logically split into several \textit{Groups}, that is sets of equal worker processes, each of which execute the same commands using same input data in the same order. In a group all worker processes run independently on different host and they do not interact with each other in any way. Current Leader Primary Process can interact with groups members using a push-based approach in order to schedule map or reduce tasks. Although, for performance reasons, not always happen, when a task is sent to the group itself, all members of the group receive it.



\subsection{Digest outputs} As explained above, in order to consider a task correct, tolerating at most $f$ faulty processes, we need $f + 1$ matching outputs have to be received. 

To validate task's output, since outputs computed by worker processes can be very large, in order to avoid pointless additional network traffic, we have adopted an approach according to which primary process fetches and compares outputs \textit{digests} from all workers within a group; this design allows us to increase system's performance.

\subsection{Crash failure detection} To manage and detect workers processes crash faults we have use some features of Apache Zookeeper. 

As known, Apache ZooKeeper has the notion of \textit{ephemeral nodes}, that is special znodes which exists as long as the \textit{session} that created the znode is active. When session expiration occurs, Zookeeper cluster will delete all ephemeral nodes owned by that session and immediately notify all connected clients of the change.

When any client establishes a session with a Zookeeper cluster, a time-out value is used by the cluster to determine when the client's session expires. Expirations usually happens when the cluster does not hear from the client within the specified session time-out period (i.e. no heartbeat).

Notice that session is kept alive by requests sent by client, therefore is critical that client will send PING request to keep the session alive. This PING request not only allows the ZooKeeper server to know that the client is still active, but it also allows the client to verify that its connection to the ZooKeeper server is still active.

Using this design, is very easy to keep check the status of worker processes.

\subsection{Deferred execution} We believe that there is no point in always executing $2f + 1$ replicas for each task to usually obtain the same result. 

To minimize both the number of copies of tasks executed and the overhead due to network traffic, saving also some energy, we have adopted a design called \textit{Deferred Execution}: according to that solution current primary starts only $f + 1$ replicas of the same task, checking if they all return the same result. If a time-out elapses, or some returned results do not match, more replicas (up to $f$) are started, until there are $f + 1$ matching replies. 

Supposing that Byzantine faults are uncommon, this design reduce the overhead introduced by the basic scheme.

\subsection{Data locality awareness}

Since moving data repeatedly among nodes is bottleneck, to improve MapReduce performance, we have adopt a design aware of data locations and sizes in order to mitigate network traffic.

Our solution is based on a basic principles which states that "\textit{moving computation towards data is cheaper than moving data towards computation}". When the map phase is fully done, all the network locations of the feeding nodes of every reducer will be known


Moving data repeatedly to distant nodes is becoming the bottleneck [23]. In this paper we rethink reduce task scheduling in Hadoop and suggest making Hadoop’s reduce task scheduler aware of partitions’ network locations and sizes in order to mitigate network traffic. 

A key question is how to schedule reduce tasks at Task Trackers so as to diminish shuffled data and improve MapReduce performance. One of Hadoop’s basic principles is: moving computation towards data is cheaper than moving data towards computation. Such a principle is employed by Hadoop when scheduling map tasks but bypassed when scheduling reduce tasks. MapReduce is aware of the network locations of splits (inputstomappers)andleveragessuchinformationtoschedule mappers nearby splits. In contrast, MapReduce is oblivious to the network locations of partitions (inputs to reducers) and does not schedule reducers nearby partitions. Thus, similar to map task scheduling, we suggest making MapReduce aware of partitions network locations in order to apply locality to reduce task scheduling



As we will be able to explain later, this design allows us to exploit data locality to increase system performance; in fact, when 





















 by a set of distributed processes:

\begin{description}
\item[Client Process] the clients that request the execution of jobs composed by map and reduce tasks

\item[Leader Primary Process] It manages the execution of word-count jobs received from clients coordinating Worker Nodes 

\item[Backup Primary Process] It manages the execution of word-count jobs received from clients coordinating Worker Nodes 

\item[Worker Process] A Worker Process executes map and reduce task scheduleted by current Leader Primary Process. In order to achieve fault tolerance, any Worker Process must be run indepently on different host. In our implementation, each process run on independent Amazon EC2 server

The Mesos master stores information about the active tasks and registered frameworks in memory: it does not persist it to disk or attempt to ensure that this information is preserved after a master failover. This helps the Mesos master scale to large clusters with many tasks and frameworks. A downside of this design is that after a failure, more work is required to recover the lost in-memory master state.



\item[Worker Group] All system's nodes in which worker process are running are logically split into several \textit{Groups}, that is sets of equal worker processes, each of which execute the same commands using same input data in the same order. In a group all worker processes run independently on different host and they do not interact with each other in any way. Current Leader Primary Process can interact with groups members using a push-based approach in order to schedule map or reduce tasks. Although, for performance reasons, not always happen, when a task is sent to the group itself, all members of the group receive it.



The key property that all groups have is that when a
message is sent to the group itself, all members of the group receive it.


 primary coordinates all write
operations


 In other words, we can replicate processes and organize them into a group to replace a single (vulnerable) process with a (fault tolerant) group.



 When a task is  for work is generated, either
by an external client or by one of the workers, it is sent to the coordinator.





as  a set of Task- Trackers that execute tasks
\end{description}


























 

\subsection{The algorithm}

In order to achieve 
A simplistic solution to make MapReduce
Byzantine fault-tolerant given the system model would be the
following. First, the JobTracker starts 2f + 1 replicas of each
map task in different servers and TaskTrackers. Second, the
JobTracker starts also 2f + 1 replicas of each reduce task.
Each reduce task fetches the output from all map replicas,
picks the most voted results, processes them and stores its
output in HDFS. In the end, either the client or a special task
must make the vote of the outputs to pick the most voted. An
even more simplistic solution would be to run a consensus,
or Byzantine agreement between each set of map task repli-
cas and reduce task replicas. This would involve even more
replicas (typically 3f + 1) and more messages exchanged.




\subsection{Crash failure detection}

\subsection{Deferred execution} As known, arbitrary faults are very hard to detect and manage 



Deferred execution. Crash faults are detected by the previ-
ously existing Hadoop mechanisms, and arbitrary faults are
uncommon, so there is no point in always executing $2f + 1$ replicas to usually obtain the same result.




By default, current leader primary process starts only $f + 1$ replicas of the same task, then wait results checking if they all return the same result. If a timeout elapses, or some returned results do not match, more replicas (up to $f$) are started, until there are $f + 1$ matching replies.



In the best case, without Byzantine faults, only f + 1 replicas are started. If arbitrary faults are uncommon, we have a < f + 1 replica started reducing the overhead 


\bibliographystyle{ACM-Reference-Format}
\bibliography{sample-base}

\appendix

\end{document}
\endinput

